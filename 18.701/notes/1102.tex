\section{November 2, 2022}

\subsection{Sylow's Theorem}

According to Prof. Cohn, the correct way to prounounce Sylow is ``See-low", at least for mathematicians. 

Here is some motivation for his theorem(s). Whenever $H$ is a subgroup of a finite group $G$, $\vert H\vert \mid \vert G\vert$. But there is not necessarily a subgroup with size every divisor of $\vert G\vert$. For example, $\vert A_5\vert = 60$, but there is no subgroup that has order $30$ in $A_5$. What we can do instead is prove some results about the existence of subgroups with order dividing prime factors of $\vert G\vert$. 

\begin{definition}
\deflabel

Let $G$ be finite group and $p$ a prime number such that $p^k\mid \vert G\vert$ but $p^{k+1}\nmid \vert G\vert$. Then, a \ac{Sylow $p$-subgroup} of $G$ is a subgroup of order $p^k$.

\end{definition}

Note: many sources will split this theorem into multiple theorems, but Prof. Cohn will present it as one huge mega theorem.
\begin{theorem}
\thmlabelname{Sylow's Theorem}

For any finite group $G$ and prime $p$, 
\begin{enumerate}
    \item [(1)] $G$ has a Sylow $p$-subgroup.
    \item [(2)] All the Sylow $p$-subgroups in $G$ are conjugate.
    \item [(3)] Every $p$-subgroup (any subgroup whose order is a power of $p$) is contained in some Sylow $p$-subgroup. 
    \item [(4)] The number of Sylow $p$-subgroups is $1\mod{p}$ and divides $\vert G\vert / p^k$.
\end{enumerate}
\end{theorem}

\subsection{Applications}

We won't go over any proofs in class today. We proved (1) on Problem Set $7$, and may present a different proof during lecture on Friday. Instead, we'll look through some examples to get a better sense of why this theorem is useful. 

\begin{example}
\exlabel

Let's examine $G = S_4$.
\end{example}

$\vert G\vert = 24 = 2^3\cdot 3$. 
\begin{itemize}
    \item There are $4$ Sylow $3$-subgroups in $G$, which are the $3$-cycles: $\langle (1 2 3)\rangle$, $\langle (1 2 4)\rangle$, $\langle (1 3 4)\rangle$, $\langle (2 3 4)\rangle$. Since these generators all have the same cycle structure, all of these groups are conjugate to one another. Also, $4\equiv 1\pmod{3}$ and $4$ divides $\vert G\vert/3=8$.
    \item There are $3$ Sylow $2$-subgroups in $G$, which are the $4$-cycles crossed with the $2$-cycles (i.e., $D_4$): $\langle (1 2 3 4), (1 3)\rangle$, etc. As before, conjugation works, and also, $3\equiv 1\pmod{2}$ and $3$ divides $\vert G\vert/8=3$.
\end{itemize}

\begin{example}
\exlabel

Let's examine $G = GL_2(\mathbb{F}_p)$.
\end{example}

$\vert G\vert = p(p-1)^2(p+1)$. One Sylow $p-$subgroup in $G$ is $\left\{
\begin{pmatrix}1 & x \\ 0 & 1\end{pmatrix} : x\in \FF_p\right\}$. This can be generalized to $GL_n(\FF_p)$ (i.e., the upper triangular matrices with ones along the diagonal).

Now, let's explore more arbitrary groups.

\begin{example}
\exlabel

Consider any $G$ with $\vert G\vert = 15 = 3\cdot 5$. 
\end{example}

Let $n_p$ be the number of Sylow $p$-subgroups. $n_3\equiv 1\pmod{3}$ and $n_3\mid 5 \implies n_3=1$. $n_5\equiv 1\pmod{5}$ and $n_5\mid 3\implies n_5=1$. Therefore, there is a unique Sylow $3$-subgroup, $H$, and a unique Sylow $5$-subgroup, $K$ in $G$.

$H$ and $K$ must be normal, since conjugating them with any element in $G$ preserves their order, and they are both unique. 

The normality of at least one of these groups is enough to imply that their set product $HK = \{hk : h\in H, k\in K\}$ is itself a subgroup of $G$. $HK$ is closed, since \[hkh'k' = \underbrace{hh'}_{\in H}\underbrace{(h')^{-1}kh'}_{\in K}k',\] using the fact that $K$ is normal. Also, every element has an inverse: \[(hk)^{-1} = k^{-1}h^{-1} = h^{-1}\underbrace{hk^{-1}h^{-1}}_{\in K},\] again using the fact that $K$ is normal. 

We further have that $3\mid \vert HK\vert$, since $H$ is a subgroup of $HK$, and $5\mid \vert HK\vert$, since $K$ is a subgroup of $HK$. This implies $\vert HK\vert \mid 15$, which means $HK = G$.

\begin{theorem}
\lemlabel 

$H$,$K$ are normal and $H\cap K = \{1\}$. Then $hkh^{-1}k^{-1} = 1$, for all $h\in H$, $k\in K$ (i.e., $hk = kh$, so multiplication is abelian). 
\end{theorem}

\begin{proof}
$\underbrace{(hkh^{-1})}_{\in K}k^{-1}\in K$, since the first conjugate is in $K$. But also, $h\underbrace{(kh^{-1}k^{-1})}_{\in H}\in H$, since the second conjugate is in $H$. Therefore, this product is in $H\cap K$, so it must be $1$.
\end{proof}

The intersection of $H$ and $K$ must be trivial, since they have different prime orders. Therefore, using the lemma, multiplication in $G$ is abelian, so $G=HK$ implies $G\cong H\times K \cong C_3\times C_5$.

\begin{example}
\exlabel 

Let's examine any group $G$ with $\vert G\vert = 2\cdot 5$.
\end{example}

Since $n_2\equiv 1\pmod{2}$ and $n_2\mid 5$, $n_2=1$ or $n_2=5$. Since $n_5\equiv 1\pmod{5}$ and $n_5\mid 2$, $n_5=1$. 

Let $H,K$ be a Sylow $2$-subgroup in $G$, and the Sylow $5$-subgroup in $G$, respectively. $K$ is normal because it is unique. $HK$ is subgroup, and $G = HK$ using the same arguments as before. Let's denote $H = \langle x\rangle$ with $x^2=1$, and $K = \langle y\rangle$ with $y^5=1$. Then $xKx^{-1} = K \implies xyx^{-1} = y^n$ for some $n\in \{1,2,3,4\}$ (if $n$ could be $0$, then $y$ would be the identity, which is not possible).

This implies $G = \{x^iy^j : 0\leq i\leq 1, 0\leq j\leq 4\}$, with $x^2=1$, $y^5=1$, $yx = xy^n$. This completely determines the multiplication table, since it tells us how to swap $x$ and $y$ and reduce any product to the form $xy^j$.

It looks like we have $4$ choices for $G$, but we can restrict $n$ even further. Given $xyx^{-1}=y^n$, we have $y = x^2yx^{-2} = x(xyx^{-1})x^{-1} = (xyx^{-1})^n = (y^{n})^n = y^{n^2}$, so $n^2\equiv 1\pmod{5}$. So it turns out that $n\neq 2,3$ and we must have $n\in \{1,4\}$. This reduces the number of choices we have for $G$ down to $2$.

When $n=1$, we have $xy=yx$, so $G$ is abelian and $G\cong C_2\times C_5$. In this case, $n_2=1$ and $n_5=1$ (which is the same case that we had in the previous example). When $n=4$, $xy = y^{-1}x$, this matches the multiplication rule for the dihedral group, so $G\cong D_5$. In this case, $n_2=5$ ($s$, $sr$, $sr^2$, $sr^3$, $sr^4$) and $n_5 = 1$.

\begin{example}
\exlabel 

Let's generalize further. Consider any group $G$ with $\vert G\vert = pq$ for primes $p,q$ and $p < q$.
\end{example}

We have $n_q\equiv 1\pmod{q}$ and $n_q\mid p$, implying $n_q=1$, so there is one unique Sylow $q$-subgroup. As before, this subgroup is normal. We also have $n_p\equiv 1\pmod{p}$ and $n_p\mid q$. If $q\not\equiv 1\pmod{p}$, this implies $n_p=1$, giving a unique Sylow $p$-subgroup. Using the same arguments that we used in Example $22.5$, this gives us that $G\cong C_p\times C_q$.

Now consider when $q\equiv 1\pmod{p}$. In this case, we have $n_p=1$ or $n_p=q$. Using the same arguments that we used in Example $22.7$, this gives us an additional possibility for $G$:
\[G = \{x^iy^j : 0\leq i\leq p-1, 0\leq j\leq q-1\},\]
with $x^p=1$, $y^q=1$, and $yx=xy^n$ for some $n\in \{1,2,\hdots, q-1\}$. As before, we can reduce our possibilities for $n$. Given $xyx^{-1} = y^n$, we have $y = x^pyx^{-p} = x(x\hdots (x (xyx^{-1})x^{-1})\hdots x^{-1})x^{-1} = y^{n^p}$, so we must have $n^p\equiv 1\pmod{q}$. 

When $n\equiv 1\pmod{q}$, $G$ is abelian, so we get $C_p\times C_q$ again. Now we claim that all $n\not\equiv 1\pmod{q}$, give the same group. The multiplicative group $\FF_q^\times$ has order $q-1$ and is cyclic. Since $p \mid q-1$, there exists a subgroup with order $p$ generated by some $n$: $\{1, n, n^2, \hdots, n^{p-1}\}$, all of which are roots to $n^p\equiv 1\pmod{q}$. But now, when we consider mapping the generator $x\mapsto x^{i}$, we have $x^{ip} = 1$ and 
\begin{align*}
    yx^i = xy^nx^{i-1} = \hdots = x^iy^{n^i},
\end{align*}
so this amounts to $n$ and $n^i$ yielding isomorphic groups for all $i$. Thus, there are only two possibilities for $G$. 
