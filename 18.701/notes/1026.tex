\section{October 26, 2022}

\subsection{$G$ acting on $G$}

Let's consider the group actions of $G$ on itself. One possible action is $(g, x)\mapsto gx$, but this is boring. A more exciting action is $(g, x)\mapsto gxg^{-1}$, otherwise known as the conjugation action. 

Let's verify that is a valid action. 
\begin{itemize}
    \item $1\times x = 1x1^{-1} = 1$
    \item $g\cdot (h\cdot x) = g\cdot (hxh^{-1}) = ghxh^{-1}g^{-1} = (gh)x(gh)^{-1} = gh\cdot x$
\end{itemize}

\begin{definition}
\deflabel

The \ac{conjugacy class} $C(x)$ for each $x\in G$ is its orbit.
\[C(x) = \{gxg^{-1} : g\in G\}\]
\end{definition}

\begin{definition}
\deflabel

The \ac{centralizer} $Z(x)$ for each $x\in G$ is its stabilizer. 
\[Z(x) = \{g\in G: gxg^{-1} = x\} = \{g\in G : gx = xg\}\]
This is also the set of elements in $G$ that commute with $x$. 
\end{definition}

By the orbit-stabilizer theorem, $\vert G\vert = \vert C(x)\vert \vert Z(x)\vert$. 

\begin{definition}
\deflabel

The \ac{center} $Z$ is the set of all elements $g\in G$ that commutes with everything. In other words, 

\[Z = \{g\in G: gx = xg \forall x\in G\} = \bigcap_{x\in G}Z(x)\]
\end{definition}

In other words, $g\in Z\iff C(g) = \{g\}\iff Z(g) = G$. Also, since elements in $Z$ commute with everything, $Z$ is a normal subgroup in $G$. 

We can provide an upper bound for the size of each conjugacy class as follows. For any $x\in G$, all powers of $x$ commute with $x$, so $\gen{x}\subseteq Z(x)$. This implies $\vert \text{order}(x)\vert \mid \vert Z(x)\vert \implies \vert C(x)\vert \leq \vert G\vert / \vert \text{order}(x)\vert$.

\subsection{Class Equations and $p$-groups}

\begin{definition}
\deflabel

The \ac{class equation} for a finite group $G$ is given by 
\[\vert G\vert = \vert C_1\vert + \vert C_2 \vert + \hdots + \vert C_m\vert,\]

where each $C_i$ are the conjugacy classes in $G$. 
\end{definition}

Let's look at a few examples of class equations for common groups.
\begin{example}
\exlabel

Consider $G = C_n$. 
\end{example}

Its conjugacy class is given by
\[n = \underbrace{1 + 1 + \hdots + 1}_n.\]

The same is true for any abelian group with order $n$. 

\begin{example}
\exlabel

Consider $G = S_3$. 
\end{example}

By the property of conjugation in permutation groups, each conjugacy class is formed by the distinct cycle structures in $G$: 
\[\{1\}, \{(12), (23), (13)\}, \{(123), (132)\}.\]

Therefore, the class equation is given by $6 = 1+2+3$. 

\begin{example}
\exlabel

Consider $G = D_n$ (the dihedral group of order $2n$). 
\end{example}

The conjugacy classes are formed by $bab^{-1}$ for all $a,b\in G$. Let's do casework on $a$ and $b$. Recall the multiplication rules for $D_n$: $x^n = 1$, $y^2 = 1$, and $yxy^{-1} = x^{-1}$. 

\begin{figure}[h]
\centering
\begin{tabular}{|c|c|c|c|c|}

    $a$ & $x^i$ & $x^i$ & $x^iy$ & $x^iy$ \\

    $b$ & $x^j$ & $x^jy$ & $x^j$ & $x^jy$ \\
    \hline
$bab^{-1}$ & $x^ix^jx^{-i}$ & $x^jyx^iyx^{-j}$ & $x^jx^iyx^{-j}$ & $x^jyx^iyyx^{-j}$\\
 & $x^j$ & $x^{-i}$ & $x^{i+2j}y$ & $x^{-i+2j}y$
\end{tabular}
\end{figure}

So $x^i$ conjugates to $x^{\pm i}$, and $x^iy$ conjugates to $x^{\pm i + 2j}y$ for any $j$. If $n$ is odd, then $x^{2j}$ can be any power of $x$. If $n$ is even, then $x^{2j}$ and $x^{1+2j}$ can only be the even and odd powers of $x$, respectively. This gives the following conjugacy classes for $D_n$: 

When $n$ is odd: 
\begin{itemize}
    \item \{1\}
    \item \{$x^i, x^{-i}$\} for all $i\neq 0$
    \item \{$x^iy$ : all $i$\}
\end{itemize}

When $n$ is even: 
\begin{itemize}
    \item \{1\}
    \item \{$x^{n/2}$\}
    \item \{$x^{i}, x^{-i}$\} for all $i\neq 0, n/2$
    \item \{$x^{2i}y$ : all $i$\}
    \item \{$x^{2i+1}y$ : all $i$\}
\end{itemize}

\begin{definition}
\deflabel

For $p$ prime, $G$ is a \ac{p-group} if $\vert G\vert = p^k$.
\end{definition}

Some examples of $p$-groups include $C_p$, $C_{p^2}$, or even
\[\left\{\begin{pmatrix}
1 & x & y \\
0 & 1 & z \\
0 & 0 & 1
\end{pmatrix} : x,y,z\in \FF_p
\right\} \in GL_3(\FF_p).\]

\begin{theorem}
\thmlabel

Every nontrivial $p$-group has a nontrivial center. 
\end{theorem}

\begin{proof}
Let $\vert G\vert = p^k$. By the orbit-stabilizer theorem, all of its conjugacy classes have order dividing $p^k$. Therefore, its class equation can be written as 
\[\vert G\vert = \sum_{i=0}^{k-1}c_i\cdot p^i,\]
where $c_i$ is the number of conjugacy classes in $G$ that have order $p^i$. This implies $p\mid c_0$; on the other hand, since the identity forms its own conjugacy class, $c_0 > 1$.

But we also know that $\vert C(x)\vert = 1\iff x\in Z$, because it means that $x$ commutes with everything. Thus, $Z\neq \{1\}$. 
\end{proof}

\begin{theorem}
\corlabel

If $\vert G\vert = p^2$ with $p$ prime, then $G$ is abelian.
\end{theorem}

\begin{proof}
By our theorem, we know that $\vert Z\vert = p$, or $\vert Z\vert = p^2$. If $\vert Z\vert = p^2$, then we are done. Otherwise, there is some $x\in Z(x)$ and $x\not\in Z$. But, since $Z\subset Z(x)\subseteq G$, $\vert Z(x)\vert = p^2$, and therefore $Z(x) = G$. But this contradicts $x\not\in Z$, so we're done.
\end{proof}

\begin{theorem}
\corlabel

If $\vert G\vert = p^2$ with $p$ prime, then $G\cong C_{p^2}$ or $C_p\times C_p$. 
\end{theorem}

\begin{proof}
If $G$ has any element with order $p^2$, then $G\cong C_{p^2}$. Otherwise, all elements (with exception to the identity) have order $p$. Pick two elements $x$ and $y$ such that $y\in G-\gen{x}$. Then $\gen{x}\cap \gen{y} = \{1\}$ implies all $x^my^n$ are distinct elements of $G$. Since there are $p^2$ elements of the form $x^my^n$, this implies $G = \gen{x,y}$. Also, since $G$ is abelian by the previous corollary, the map $\gen{x,y}\rightarrow C_p\times C_p$ given by $(x^m,y^n)\mapsto (m,n)$ is an isomorphism, so we are done. 
\end{proof}
