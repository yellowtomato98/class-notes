\section{September 12, 2022}

\subsection{Subgroups}

Let $G$ be a group, and $S\subseteq G$ any subset of $G$. 

\begin{definition}
\deflabel

The subgroup of $G$ generated by $S$ is equal to the intersection of all subgroups of $G$ containing $S$.
\end{definition}

\begin{example}
\exlabel

For any $g\in G$, when $S = \{g\}$, this subgroup is $\gen{g} = \{g^n : n\in \ZZ\}$. 
\end{example}

If $\vert\gen{g}\vert$ is infinite, then we say that the \ac{order} of $g$ is infinite. In this case, the subgroup is isomorphic to $\ZZ$ (the infinite cyclic group). Otherwise, the order of $g$ is some finite $k$, meaning that $k$ is the minimum power of $g$ that makes it equal to the identity, and $\gen{g}\cong C_k$.  

\subsection{Subgroups of $\ZZ$}
For each $d\in \ZZ$, $\gen{d} = d\ZZ$ is a subgroup of $\ZZ$. This turns out to be the only possible type of subgroup.
\begin{center}
\begin{asy}
import graph; size(3.5cm); 
usepackage("amsfonts");
pen dps = linewidth(0.7) + fontsize(10); defaultpen(dps);
pen dotstyle = black;
real sz=1.5;
real sz2=1.7;

pair A = (0,0.7);
pair B = (0,-0.7);
pair C = (1,0); 
pair D = (-1,0);
pair E = (-1,1.4);
pair F = (-1,-1.4);

label(scale(sz)*"$2\mathbb{Z}$", A);
label(scale(sz)*"$3\mathbb{Z}$", B);
label(scale(sz)*"$\mathbb{Z}$", C);
label(scale(sz)*"$6\mathbb{Z}$", D);
label(scale(sz)*"$4\mathbb{Z}$", E);
label(scale(sz)*"$9\mathbb{Z}$", F);

label(rotate(-40)*scale(sz2)*"$\subseteq$", shift(0.2,0.35)*(E--A));
label(rotate(40)*scale(sz2)*"$\subseteq$", shift(-0.15,0.27)*(D--A));
label(rotate(-40)*scale(sz2)*"$\subseteq$", shift(0.2,0.25)*(D--B));
label(rotate(40)*scale(sz2)*"$\subseteq$", shift(-0.05,0.2)*(B--C));
label(rotate(-40)*scale(sz2)*"$\subseteq$", shift(0.2,0.25)*(A--C));
label(rotate(40)*scale(sz2)*"$\subseteq$", shift(-0.1,0.25)*(F--B));

label(scale(1.5)*"$\cdots$", (-1.8,0));
\end{asy}
\end{center}

\begin{theorem}
\thmlabel

Every subgroup of $\ZZ$ has a single generator. 
\end{theorem}

\begin{proof}

Let $H$ be a subgroup of $\ZZ$. If $H = \{0\}$, then $H = 0\ZZ$. Otherwise, $H$ contains some positive integer. 

Let $d$ be the smallest positive integer in $H$, which exists by the well-ordering principle. For any $n\in H$, the remainder theorem implies that there exists $q,r\in \ZZ$ such that $n = dq+r$, with $0\leq r < d$. Since $d\in H$, and $n\in H$, $r = n-dq\in H$. But $d$ is the smallest positive integer in $H$, so $r=0\implies n\in d\ZZ \implies H\subseteq d\ZZ$. We also know $d\ZZ\subseteq H$, since $d\in H$, thus $H = d\ZZ$ has a single generator, as desired. 
\end{proof}

\subsection{Homomorphisms}

\begin{definition}
\deflabel

Given two groups $(G, \cdot)$ and $(H, \star)$, a \ac{homomorphism} from $G$ to $H$ is a map $f:G\rightarrow H$ satisfying $f(g\cdot g') = f(g)\star f(g')$ for all $g,g'\in G$. 
\end{definition}

This definition matches the definition for an isomorphism, with the exception that our map does not need to be one-to-one. In other words, an isomorphism is a one-to-one homomorphism. 

\begin{theorem}
\proplabel

$f(1_G) = 1_H$.
\end{theorem}

\begin{proof}
$f(1_G)\star f(1_G) = f(1_G)\implies f(1_G) = 1_H$ by multiplying by the inverse of $f(1_G)$ on both sides. 
\end{proof}

\begin{theorem}
\proplabel

$f(g^{-1}) = f(g)^{-1}$ in $H$. 
\end{theorem}

\begin{proof}
Since the identity is preserved under our map, $f(g)f(g^{-1})=f(gg^{-1}) = 1_H\implies f(g^{-1}) = f(g)^{-1}$.
\end{proof}

Now, let's look at some examples of homomorphisms. 

\begin{example}
\exlabel

Consider any subgroup $H\subseteq G$. 
\end{example}

The identity map $f : H\rightarrow G$ is a homomorphism, so homomorphisms generalize subgroups. $f$ is an ``inclusion map''.  

\begin{example}
\exlabel

$\ZZ$ and $C_n$. 
\end{example}

There is a homomorphism $\ZZ\rightarrow C_n$ taking $k\mapsto k\pmod n$, also called reduction mod $n$. In this lens, homomorphisms also generalize modulo arithmetic. 

\begin{example}
\exlabel

$S_n$ and $\GL_n(\RR)$.
\end{example}

Verify that the map $S_n\rightarrow \GL_n(\RR)$ with $\pi \mapsto$ (permutation matrix of $\pi$) is a homomorphism.

\begin{example}
\exlabel

The determinant is a homomorphism.
\end{example}

$\det : \GL_n(\RR)\rightarrow \RR^x$ with $A\mapsto \det A$ is a homomorphism. (Recall that $\RR^x$ is the set of non-zero real numbers, which is a group).

\begin{example}
\exlabel

The sign homomorphism. 
\end{example}

The sign homomorphism is defined as the map $\text{sgn} : S_n\rightarrow \{\pm 1\}$ with $\text{sgn}(\pi) = \det (\text{perm. matrix})$. This is a homomorphism, since 
\[\det(A_{\pi})\det(A_{\pi^{-1}})=1,\]
so all determinants are $\pm 1$. 

\begin{example}
\exlabel

Exponentiation is a homomorphism. 
\end{example}

The map $\exp: \CC\rightarrow \CC^x$ with $\exp z = e^z$ is a homomorphism (here we take $\CC$ over addition and $\CC^x$ over multiplication). $\exp$ is not isomorphic, since $e^{2\pi i} = e^0 = 1$ (the map is not injective). 