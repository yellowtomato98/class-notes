\section{October 19, 2022}
\subsection{Group Actions}

\begin{definition}
\deflabel

Given a group $G$ and set $S$, the \ac{action} of $G$ on $S$ is a function $G\times S\rightarrow S$ mapping $(g,s)\mapsto g\cdot$ satisfying 
\begin{enumerate}
    \item [(i)] $1\cdot s = s\quad \forall s\in S$
    \item [(ii)] $(gh)\cdot s = g\cdot (hs) \quad \forall g,h\in G, s\in S$ 
\end{enumerate}
\end{definition}

Intuitively, a group action can be thought of as a composition of ``functions" (elements in $g$) on elements in $S$, such that elements of $S$ are mapped to other elements of $S$.

Here are some examples:
\begin{itemize}
    \item Given a field $F$, $GL_n(F)$ acts on $F^n$ through the action $A\cdot x = Ax$.
    \item $S_n$ acts on $\{1,2,\hdots, n\}$ through the action $\pi \cdot i = \pi(i)$.
    \item $M_n$ acts on $\RR^n$ by $f\cdot x = f(x)$.
\end{itemize}

\begin{theorem}
  \proplabel
  
  An action of $G$ on $S$ implies a homomorphism $f: G\rightarrow \pi(S)$, where $\pi(S)$ denotes the set of permutations of $S$.
\end{theorem}

\begin{proof}
  Let $f(g)(s) = g\cdot s$. First, we show that $f(g)$ is an element of $\pi(s)$, which amounts to showing that $f(g)$ is a bijective map. The co-domain of $f$ is $S$ by the definition of a group action. 
  
  First, $f(g^{-1})(f(g)(s)) = f(g^{-1})(g\cdot s) = s$. Thus $f(g^{-1})f(g)$ is the identity function, implying that $f(g)$ is injective; if not, then $f(g^{-1})f(g)$ maps two elements to the same element. 
  
  Second, $f(g)f(g^{-1})(s) = s$, so $f(g)f(g^{-1})$ is also the identity function. This implies that $f(g)$ is surjective, since $f(g)f(g^{-1})$ is surjective. Thus, $f(g)$ is a bijective mapping.
  
  It remains to show that $f$ is a homomorphism:
  \begin{align*}
      f(g_1g_2)(s) &= (g_1g_2)\cdot s \\
      &= g_1\cdot (g_2\cdot s) \\ 
      &= f(g_1)f(g_2)(s).
  \end{align*}
\end{proof}

\begin{theorem}
\thmlabelname{Cayley's theorem}

If $\vert G\vert = n$, then $G$ is isomorphic to a subgroup of $S_n$.
\end{theorem}

\begin{proof}
Consider the action of $G$ on itself given by $g\cdot h = gh$. Using the above proposition, there is a homomorphism $f: G\rightarrow \pi(G)$. Then, since $\ker{G} = \{1\}$, $f$ is injective, so $G$ is isomorphic to some subgroup of $\pi(G)$, implying the result.
\end{proof}

\subsection{Orbit-Stabilizer Theorem}
Suppose $G$ acts on $S$.
\begin{definition}
\deflabel

The \ac{orbit} of $s\in S$ is the set 
\[Gs = \{g\cdot s : g\in G\}.\]
\end{definition}

In other words, the orbit of $s$ as its image over all possible elements in $G$. Orbits may overlap; for example, when $s_1$ is in the orbit of $s_2$, then $s_2$ will also be in the orbit of $s_1$, and in particular the orbits of $s_1$ and $s_2$ will be the same. 

\begin{definition}
\deflabel

The \ac{stabilizer} of $s\in S$ is the set 
\[\stab_G(s) = \{g\in G: g\cdot s=s\}.\]
\end{definition}

In other words, the stabilizer of $s$ is all elements in $G$ that fix $s$. 

\begin{definition}
\deflabel

The action of $G$ on $S$ is \ac{transitive} if $Gs=S$ for some (all) $s\in S$.
\end{definition}

\begin{example}
\exlabel

The action of $D_4$ on $\RR^2$. $D_4$ is the dihedral group of order $4$, and the possible set of actions are $90^{\circ}$ rotations of the plane, or reflections of the plane. 
\end{example}

Let $O$ be the origin. Then,
\begin{itemize}
    \item $D_4O = \{O\}$. 
    \item $\stab_{D_4}(O) = D_4$.
\end{itemize}

Let $XYZW$ be a square with center at $O$. Then, 
\begin{itemize}
    \item $D_4X = \{X, Y, Z, W\}$.
    \item $\stab_{D_4}(X) = \{1, \text{reflection}(XZ)\}$.
\end{itemize}

\begin{theorem}
\proplabel

The set of orbits partition $S$. 
\end{theorem}

\begin{proof}
Suppose $Gs_1\cap Gs_2\neq \emptyset$. Then, $g_1s_1 = g_2s_2$ for some $g_1, g_2\in G$. But then $s_1 = g_1^{-1}g_2s_2$, so $s_1\in Gs_2$, and $s_2 = g_2^{-1}g_1s_1$, so $s_2\in Gs_1$, and therefore $Gs_1=Gs_2$. In other words, if any two orbits overlap, they must be the same orbit. Since all elements of $s$ are part of their own orbit, all elements of $S$ are in some orbit, so the proposition follows. 
\end{proof}

\begin{theorem}
\thmlabel

Let $G$ act on $S$, $s\in S$, and $H = \stab_G(s)$. Then, there exists a bijection $G/H\rightarrow Gs$ mapping $gH\mapsto gs$. (Here, let $G/H$ denote the set of left cosets of $H$, not the normal quotient group). 
\end{theorem}

\begin{proof}
Let $f$ be our bijection. Then, $f$ is well-defined, since $g_1h_1=g_2h_2\implies g_1h_1s=g_2h_2s$ is always true since $h_1s = h_2s = s$ ($H$ is the stabilizer). Therefore, $g_1H = g_2H\implies g_1s = g_2s$.

$f$ is surjective, since every element of $Gs$ is $gs = f(gH)$. If $f(g_1H) = f(g_2H)$, then 
\begin{align*}
    g_1s &= g_2s \\
    \implies g_2^{-1}g_1s &= s \\
    \implies g_2^{-1}g_1&\in H \\
    \implies g_1&\in g_2H.
\end{align*}
Applying this symmetrically implies $g_1H = g_2H$, so $f$ is injective, and the result follows.
\end{proof}

\begin{theorem}
\thmlabelname{Orbit-Stabilizer Theorem}

For all $s\in S$, 
\[\vert G\vert = \vert Gs\vert \cdot \vert \stab_G(s)\vert.\]
\end{theorem}

\begin{proof}
Follows from the previous Theorem.
\end{proof}

\begin{example}
\exlabel

Consider the group $G$, the set of rotations of a cube, acting on $S$. We can find $\vert G\vert$ in three different ways by letting $S$ equal the set of faces, edges, or vertices of a cube.

\begin{itemize}
    \item If $S$ is the faces of a cube, then $\vert G\vert = 6\cdot 4$, because for any $s\in S$, there are $6$ ways to map $s$ to another face ($\vert Gs\vert$) and $4$ rotations preserving that face ($\vert \stab_G(s)\vert$).
    \item If $S$ is the vertices of a cube, then $\vert G\vert = 8\cdot 3$, because for any $s\in S$, there are $8$ ways to map $s$ to another vertex and $3$ rotations preserving that vertex.
    \item If $S$ is the edges of a cube, then $\vert G\vert = 12\cdot 2$, because for any $s\in S$, there are $12$ ways to map $s$ to another edge and $2$ ways to rotate the cube to preserve that edge, by flipping its vertices.
\end{itemize}
\end{example}

\subsection{Burnside's Lemma}

Addendum: This was not covered in lecture, but the Orbit-Stabilizer Theorem implies Burnside's Lemma. 

\begin{theorem}
\thmlabelname{Burnside's Lemma}

Let $s\in S$ be a fixed point for $g\in G$ if $g\cdot s = s$. Let $k$ be the number of orbits $Gs$. Then, the average number of fixed points for any $g\in G$ is equal to $k$.
\end{theorem}


\begin{proof}
Let $S_i$ denote the $i$th orbit.
\begin{align*}
    \sum_{g\in G}\vert \{s\in S : g\cdot s = s\}\vert &= \sum_{s\in S}\vert \{g\in G: g\cdot s = s\}\vert\\
    &= \sum_{i=1}^k \sum_{s\in S_i}\vert\stab_G(s)\vert \\
    &= \sum_{i=1}^k \sum_{s\in S_i}\frac{\vert G\vert}{\vert S_i\vert} \\
    &= k\vert G\vert,
\end{align*}
where the third equality follows from the orbit-stabilizer theorem.
\end{proof}

\begin{example}
\exlabel 

Consider the group $G$, the set of rotations of a square, acting on $S$, the set of all possible colorings of the square with $n$ colors. In order to count the number of distinct ways to color the square, where colorings that can be obtained via rotation are considered the same, we want to find the number of orbits. 
\end{example}

$G = C_4$ consists of the identity, a rotation by $\pm 90^{\circ}$, and a rotation by $180^{\circ}$. The identity fixes $n^4$ elements, the rotations by $\pm 90^{\circ}$ fixes $n$ elements, and the rotation by $180^{\circ}$ fixes $n^2$ elements. Thus, the number of orbits is $(n^4+n^2+2n)/4$, which is also the number of colorings up to rotation.





  