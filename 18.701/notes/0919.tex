\section{September 19, 2022}

\subsection{Quotient Groups}

Recall from last lecture: given group $G$ and subgroup $H\subseteq G$, 
\[H\text{ normal}\iff gH=Hg\quad \forall g\iff gHg^{-1}=H \quad \forall g\iff ghg^{-1}\subseteq H \quad \forall g,h.\]

The last equivalence follows, because the last statement implies $gH\subseteq Hg$, and you can flip $g$ and $g^{-1}$ to get $gH\supseteq Hg$, so $gH=Hg$. 

We showed last time that all kernels are normal. Most subgroups aren't normal. Given $H$ normal, does there exist a homomorphism $f: G\rightarrow G'$ such that $\ker(f) = H$? The answer is yes, which we'll show today. 

\begin{theorem}
\claimlabel 

If $f$ is surjective, then $G'$ is uniquely determined by $G,H$ up to isomorphism. 
\end{theorem}

The idea is that $G'$ is determined by left cosets of $H$ in $G$. If $H$ is the kernel, then $gH$ is the preimage of $f(g)$ in $G$. In other words, each coset corresponds to a unique element of $\im (f)$. 

Here's how we can turn cosets into a group.
\begin{definition}
\deflabel

Given cosets $C_1$ and $C_2$, let their product be 
\[C_1C_2 = \{c_1c_2 : c_1\in C_1, c_2\in C_2\}.\]
\end{definition}

This way of multiplying cosets would be terrible if $H$ was not normal. Given that $H$ is normal, then
\[g_1H\cdot g_2H = g_1(Hg_2)H = g_1g_2 H,\]
which is also a coset. Note that the second equality only works when $H$ is normal. Since the cosets are closed under this multiplication, we can treat them like a group. 

\begin{definition}
\deflabel

If $G$ is a group with normal subgroup $H$, the \ac{quotient group}
\[G/H = \{gH : g\in G\},\]
with coset multiplication. 
\end{definition}

This is a group:
\begin{itemize}
    \item Closure, since $H$ is normal
    \item Associative, since $(g_1Hg_2H)g_3H = (g_1g_2g_3)H = g_1H(g_2Hg_3H)$.
    \item Identity $H$
    \item Inverse, $(gH)^{-1} = g^{-1}H$. 
\end{itemize}

\begin{definition}
\deflabel

The \ac{canonical map} $\pi : G\rightarrow G/H$ given by $g\mapsto gH$. 
\end{definition}

\begin{theorem}
\thmlabel

The canonical map $\pi$ is a surjective homomorphism with $\ker(\pi)=H$. 
\end{theorem}

\begin{proof}
By definition, $\pi$ is surjective. $\pi$ is also a homomorphism: 
\[\pi(g_1g_2) = g_1g_2(H) = g_1Hg_2H = \pi(g_1)\pi(g_2).\]

Now $\pi(g)=H\iff gH=H\iff g\in H$, so $\ker(\pi)=H$. 
\end{proof}

\subsection{First Isomorphism Theorem}

\begin{theorem}
\thmlabelname{First Isomorphism Theorem}

Given a surjective homomorphism $f : G\rightarrow G'$ with $\ker(f) = H$, there exists a unique isomorphism $\ov{f} : G/H\rightarrow G'$ s.t. $f = \ov{f}\circ \pi$. 
\end{theorem}

In other words, this diagram:
\begin{center}
\adjustbox{scale=1.1}{%
    \begin{tikzcd}[column sep=tiny, row sep=small]
        G \arrow[rr, "f"] \arrow[rdd, "\pi"'] & & G'\\
        \\
        & G/H\arrow[ruu, "\overline{f}"']
    \end{tikzcd}
}
\end{center}

is a commutative diagram (different paths in the diagram leads to the same mapping). 

\begin{proof}
$\ov{f}$ is uniquely determined by $\ov{f}(gH) = f(g)$. This implies that $\ov{f}$ is well-defined and injective, since $f(g_1H)=f(g_2H)\iff f(g_1)=f(g_2)\iff g_1H=g_2H$ ($H=\ker(f)$). $\ov{f}$ is surjective since $f$ is surjective.

Also, $\ov{f}$ is a homomorphism since $f$ is a homomorphism: 
\[\ov{f}(g_1Hg_2H) = f(g_1g_2)=f(g_1)f(g_2) = \ov{f}(g_1H)\ov{f}(g_2H),\]
so $\ov{f}$ is an isomorphism and we're done. 
\end{proof}

This theorem gives us a nice way to break down any homomorphism $f: G\rightarrow H$ that exists between arbitrary groups $G$, $H$. Specifically, $f$ is surjective when you restrict the mapping to $\im(f)$, so $f$ is the composition of three mappings: 
\begin{itemize}
    \item quotient $\pi: G\rightarrow G/\ker(f)$
    \item isomorphism $\ov{f}: G/\ker(f)\rightarrow \im(f)$
    \item inclusion $\im(f)\rightarrow H$,
\end{itemize}
where the inclusion just maps everything to itself and takes care of the fact that the codomain of the original mapping was to all of $H$. 

\begin{center}
\adjustbox{scale=1.1}{%
    \begin{tikzcd}[column sep=tiny, row sep=small]
        & G \arrow[ldd, "\pi"'] \arrow[rr, "f"] \arrow[rdd, "f"'] & & H\\
        \\
        G/H \arrow[rr, "\ov{f}"'] & & \im(f)\arrow[ruu, "\text{inc}"'] &
    \end{tikzcd}
}
\end{center}

Here are some examples of the first isomorphism theorem giving useful equivalencies for common groups.  
\begin{example}
\exlabel 

$\ZZ/n\ZZ\cong C_n$. 
\end{example}

Recall from last lecture the homomorphism $f: \ZZ\rightarrow C_n$ mapping $1$ to any generator. $\ker(f) = n\ZZ$, hence the correspondence between the cosets and $C_n$. 

\begin{example}
\exlabel

$GL_n(\RR)/SL_n(\RR)\cong \RR^x$.
\end{example}

The homomorphism $\det : GL_n(\RR)\rightarrow \RR^x$ has kernel $\SL_n(\RR)$. 

\begin{example}
\exlabel

$S_n/A_n\cong \{\pm 1\}\cong C_2$.
\end{example}

